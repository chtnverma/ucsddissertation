\subsection{Related work} 
\label{sec:relatedwork}
We discuss below the related work on video classification. Since the keyword selection based approach requires optimizing for two objective functions, we also provide a discussion on the related work on multi-objective optimization. \\

\noindent \textbf{Video classification: }\\ 
A significant amount of work has been done to address the problem of video classification. Such work can be looked at on the basis of two dimensions -– modalities used for classification, and approaches to obtain labeled training videos.  While the focus of our work is on obtaining training videos, we first briefly describe video classification approaches in terms of modalities used, including our approach, and then discuss approaches to obtain training data, contrasting our approach from others. 

A characteristic property of web videos is that they have rich information in several modes -- audio, visual, textual, and social network being the most common ones. Methods such as \cite{song2010taxonomic,wang2010youtubecat,zhang2011improving,ramachandran2009videomule,yang2007multi} present multi-modal techniques for classification of web videos. Others such as \cite{schindler2008internet,chen2010effective} classify videos using only the audio-visual information in the videos, while \cite{chen2010web,wu2010data} approach classification of web videos by treating them as text documents. A detailed survey on video classification is provided in \cite{brezeale2008automatic}. In our work, we classify web videos on the basis of the contextual information surrounding them, such as the title, keywords, and description. This is because text-based classification approaches are computationally much less expensive than multimedia features-based classification approaches, and as shown in existing literature as well as in Section~\ref{sec:expt}, offer good classification performance.  

With respect to the approaches to obtain labeled training videos, \cite{chen2010web,wu2010data,ramachandran2009videomule}   obtain training videos that are labeled according to categories used by YouTube~\cite{Youtube}. Hence, such approaches cannot be used for classification of web videos to arbitrary set of categories, which is the focus of this paper. Approaches such as \cite{schindler2008internet,yang2007multi,song2010taxonomic}  utilize training videos that are labeled manually. Recently, techniques have been developed \cite{wang2010youtubecat,zhang2011improving} which expand the set of training videos starting from a set of manually labeled videos. With the help of social network structure of the video sharing website, co-watched videos, or text-based classifiers, \cite{wang2010youtubecat,zhang2011improving} increase the number of training videos in a semi-supervised fashion. However, manual labeling requires human experts to go through at least a part of the video, and come up with a label. The labeling process is prone to human errors and inconsistencies, and more critically, is not scalable to large sizes of training data, especially given the enormous scale of web videos \cite{Neilson2011}. Contrary to these approaches, we propose a framework that does not require any manual effort to obtain training videos, even for any arbitrary set of categories desired by a personalization application.\\

\noindent \textbf{Multi-objective optimization: } \\
In general, for multi-objective optimization problems, there does not exist a single solution that optimizes all objectives. \cite{Marler04} surveys the different approaches that are adopted to solve multi-objective optimization problems. Certain approaches require apriori knowledge of the preferences for the different objective functions. In order to utilize these preferences, a global objective function can be defined based on the individual objective functions and by using the preferences as numerical weights, as is shown in \cite{Papalambros96,Bridgman22,Koski87}. For example, a common approach to multi-objective optimization is to combine the individual objective functions through a weighted sum method as shown in \cite{Steuer89}. 

Other works embed the notion of preferences into their methodology, for example by ranking the objective functions in the order of their relative importance \cite{YoonHwang95,Hertz94} or by modifying the bounds of the individual optimization problems \cite{HwangMasud79,Haimes71}. Such works however require the apriori knowledge of the preferences of the multiple objectives, as obtained by a human decision maker. 
%Such human involvement makes the keyword selection procedure non scalable, subjective and time consuming. 
In the context of our problem of selecting keywords to obtain training videos, such relative preferences of different objective functions is unavailable and hence these approaches cannot be applied. 

A different line of works approach multi-objective optimization problems by first obtaining a set or a representation of \textit{pareto optimal} solutions for the problem, and then employing human judgment to decide the best solution. Pareto optimal solutions are alternative solutions for the multi-objective optimization problem that are optimal in the wider sense that no other solutions in the search space are superior to them when all the objectives are considered \cite{SumanAnnealing05}. Refer to \cite{Marler04} for the mathematical definition of pareto optimality. Thus, works such as \cite{Messac02,Martinez01,Messac01} focus on finding the set of pareto optimal solutions or their representation to serve as a palette of solutions for the human decision-maker. Varying the weights given to different objective functions is a common approach to determine the pareto optimal solutions or their subset \cite{Marler04}. Such approaches however can be extremely inefficient as they require solving the weighted formulations several times \cite{Marler04}. \cite{Hansen97} works with a set of solutions searching for Pareto optimal solutions in parallel. In order to find the best candidate in the neighborhood of each current solution, the goodness is taken as the weighted average of the objective functions. The weights are dynamically updated so that unexplored regions of solution space get more preference. Certain works such as \cite{Suman05,Smith04,Bandyopadhyay08} approach the search for pareto solutions using simulated annealing framework based on an energy function for states. These energy functions are defined based on the number of solutions that dominate or are dominated by a particular solution. See \cite{Bandyopadhyay08} for a definition of domination. Such approaches require the computation of the objective functions for all or a subset of the possible solutions. While these approaches may work for certain domains \cite{Suman05,Smith04,Bandyopadhyay08}, they would be infeasible to apply to domains when the discrete solutions are subsets of a larger set and thus the number of solutions is extremely large. As shown in Section~\ref{sec:aao}, for our problem of selecting keywords, the total number of subsets can easily be of the order of $10^{30}$ and hence such approaches would be infeasible. 
In addition, in order to use the solution in a practical setting, such approaches would still require a decision maker to choose the best solution based on his/her human judgment. Compared to these, we propose two approaches LCPD and AAO to select keywords to obtain training videos. In the former, we heuristically combine two objectives linearly and select keywords one by one without computing the objective function for all subsets. In the latter, we balance the two objective functions in our formulation (Section~\ref{sec:aao}) by alternating between algorithms that optimize for each objective individually, in a simulated annealing framework. As a result of this, for AAO it is not required to provide preferences for the individual objective functions and the approach yields a solution that is convenient trade-off between the two objective functions. We empirically demonstrate the effectiveness of the two proposed approaches for selecting keywords to obtain the training videos for a category. Furthermore for AAO, in order to reduce the time taken for the annealing based approach to converge, we propose an adaptive technique to update the annealing constraint parameter, leading to substantial reduction in the convergence time. 

This work is based on \cite{VermaWI13} which provides an approach to automatically obtain training videos. We refer to the techniques proposed in \cite{VermaWI13} as LCPD, and in addition we propose an annealing based alternating optimization approach (AAO) that does not depend on parameters that may require tuning based on human input as in \cite{VermaWI13}. Further, we propose adaptive variation of AAO to make it more efficient, and provide performance comparisons with \cite{VermaWI13}. 

For multi-class, single-label classification of web videos, we first discuss the desired properties of training videos that can lead to high performance of trained classification models. This is done in Section~\ref{sec:desiredpropdata}. Section~\ref{sec:overview} provides an overview of our framework of identifying keywords to retrieve training videos with the desired properties. It also provides a discussion on the two objective functions to select keywords. Section~\ref{sec:selectionSRK} discusses the two approaches - LCPD and AAO. Section~\ref{sec:expt} details the experimental set-up and presents performance results. Section~\ref{sec:conclusion} concludes.
