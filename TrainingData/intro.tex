\section{Introduction}
\label{sec:intro}
Over the past few years, there has been a steady rise in the number and popularity of personalization applications available on the Internet. These include applications based on personalized advertisements, content recommendation systems, social network connection suggestions, and several others, that attempt to understand the preferences of users. Personalization applications have been traditionally based on learning user preferences through queried keywords and viewed articles~\cite{mobasher2000automatic}. The last few years have also witnessed a tremendous increase in viewing and sharing of web videos (such as on YouTube\cite{Youtube}), with significant increases in unique viewers, total streams viewed, number of streams per viewer, and the time per viewer~\cite{Neilson2011}. Given their unique characteristics, web videos offer a tremendous potential for understanding user preferences.

User preferences can be inferred based on the types/categories of web videos seen. Such videos are generally organized at video sharing websites on the basis of labels that the video uploaders choose from among a set of common categories that are used by such websites. Examples of such common categories include Comedy, Music, People, Entertainment, Pets, Science, etc. On the other hand, the categories of interest to personalization applications may be arbitrary, and quite different from the above common categories. Consider a department store (such as Sears or Walmart) that might want to offer promotional coupons to buyers. Knowing whether a person (a buyer) has interest in product specific categories like fitness equipment, clothing items, or baby products would be of high interest to the department store, as compared to knowing whether he/she is interested in the common categories mentioned above. A movie recommendation system would like to learn if a viewer prefers action, horror, or comedy movies. Categorizing viewed videos and understanding user preferences in terms of the common categories used by video sharing websites might not be useful for different personalization applications. In addition to the above observation, it should be noted that different personalization applications are interested in understanding user preferences with respect to very different sets of categories, as shown by the above examples. It is clearly not sufficient to use a common set of categories for every personalization application, as the categories of interest for one application might be irrelevant and useless for another.

This calls for techniques to classify viewed web videos, and hence estimate user preferences, in terms of any arbitrary set of categories appropriate for a given personalization application. Various modes of information (such as audio, visual, textual and social network) can be employed to assist in the classification of web videos. Classifiers employed for this task have the inherent requirement of training videos labeled to the set of categories as desired by the personalization application. Since the set of categories suitable for a personalization application might be very different from the common categories used by video sharing websites, training videos labeled according to the required set of categories are often unavailable. Our work addresses this requirement of obtaining training videos labeled as per the required set of categories, which are not necessarily the categories commonly associated with web videos. We propose a fully automated framework to obtain training videos with properties that can lead to high performance of trained classification models. To achieve the above, the proposed framework neither relies on labels associated with online videos, nor requires any manual labeling of videos. Instead, we develop approaches to select keywords based on their suitability to retrieve high quality training videos for a specified set of categories. Such a methodology requires the consideration of two opposing objective, namely proximity and diversity. In order to select the keywords, we first present a tunable approach  (LCPD) based on the Linear Combination of Proximity and Diversity. While such an approach can give good classification performance, it requires the tuning of a parameter in order to obtain optimal performance, which may require manual effort. We thus also propose an approach using Annealing based Alternating Optimization (AAO) where we balance between the two objectives such that the final solution is a trade-off between the two. Complexity, convergence and correctness of the proposed algorithms are presented, along with experiments over several sets of categories. 
