\section{Conclusions}
\label{sec:conclusion}

We have proposed a fully-automated framework to obtain high quality training videos for any arbitrary set of categories, without the need for any manual labeling that is needed by most related approaches. We analyze properties of training data that lead to high performance of the trained classifier. Based on the above properties, we propose approaches for selecting keywords to retrieve training videos, on the basis of their proximity to the categories of interest, and the resulting diversity in the training data. The first approach (LCPD) leads to high classification accuracy, although requires a parameter representing preference for defined objective functions. The parameter can be obtained through manually provided preferences for the objective functions, or by using manually labeled validation videos for tuning. In order to avoid the manual effort, we also provide an annealing based alternating optimization framework (AAO) and propose its adaptive variant (Adapt AAO) to select keywords. Such an approach does not require the articulation of preferences in parameterized form, with the trade-off of some loss in classification accuracy as compared to the LCPD approach. Experimental results on several sets of categories show the effectiveness of the training videos obtained by the proposed approaches, hence making classification of videos watched by users to arbitrary set of categories feasible. Consequently, this work may enable new personalization applications by enabling identification of user preferences in a set of categories relevant to the application.

\section{Acknowledgements} 

We thank the anonymous WI 2013 and IEEE Access reviewers for their feedback and comments on the work. The authors are thankful to Dr. Nitesh Shroff for insightful discussions. The research discussed in this chapter was supported by the UCSD Center for Wireless Communication and the UC Discovery Grant program. 

Chapter 2, in part, contains material as it appears in the Proceedings of the 13th International Conference on Web Intelligence (WI'13). ``Fully Automated Learning for Application-Specific Web Video Classification''. Chetan Verma, Sujit Dey. The dissertation author was the primary investigator and author of this paper. 

Chapter 2, in part, contains material as it appears in IEEE Access. ``Methods to Obtain Training Videos for Fully Automated Application-Specific Classification''. Chetan Verma, Sujit Dey. The dissertation author was the primary investigator and author of this paper. 
