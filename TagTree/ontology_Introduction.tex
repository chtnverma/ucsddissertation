
\section{Introduction} \label{sec:Introduction}
The consumer electronic revolution and the Internet have led to the availability of vast amounts of data including multimedia data such as images and videos. A significant fraction of such data is user generated content, in the form of pictures and videos uploaded onto sites such as Facebook~\cite{Facebook}, Flickr~\cite{Flickr} and YouTube~\cite{Youtube}. Owing to the fact that there are minimal requirements when uploading the content and that mobile uploads are on the rise, users rarely add any extra information such as a textual description to the content. At best, most images and videos are {\em tagged} with certain keywords. As these keywords or tags are sometimes applied to entire albums of images or videos at once, or applied in error, the information provided by such tags is quite noisy. %We define noisy tags as ones that would not be associated with a given image, as per a trained human annotator, without any contextual information in addition to the visual information provided in the image. 
\begin{figure}[htp]
\centering
\begin{tabular}{p{3cm} p{3cm} p{3cm}}
\centering
\epsfig{width=1.5cm,height=2.1cm,figure=TagTree/figures/4670326818_bf12bf1525.eps} &
\epsfig{width=1.5cm,height=2.1cm,figure=TagTree/figures/252474171_7f272001c5.eps} &
\epsfig{width=2.1cm,height=1.5cm,figure=TagTree/figures/5835556089_812a272a59.eps}\\
(a) `animal', {\bf `car'}  & 
(b){\bf `wedding'}, `mushroom' & 
(c) `farms', {\bf `snow'} 
\end{tabular}
\caption{Examples of incorrect tags given by users on {\tt www.flickr.com}. (a) An image of a `cat' tagged as `car', which most likely is a spelling mistake, (b) an image of a `mushroom' also tagged as `wedding' and (c) an image of a `goat' tagged with `snow'. The Flickr owner and photo ids of these images are (8656572@N04,4670326818), (35468147887@N01,252474171), (39405339@N00,5835556089) respectively.}
\label{fig:flickrnoise}
\end{figure}
%\hspace{-2cm}
Some examples of images having incorrect tags (as per human experts) are shown in Fig.~\ref{fig:flickrnoise}. The massive scale of data and the lack of useful metadata makes it difficult for users to access data that may be of interest to them~\cite{DeepaFolkso14},\cite{ShuhuiAuthor15},\cite{ZhengRecom10}. 
%. This is commonly refered to as information overload in literature.


The social tagging at the above mentioned data sharing websites creates a {\em Folksonomy}~{\cite{HsiehCollab09},\cite{SunLang11},\cite{HyunwooFrame14}} which mitigates the information overload to some extent by creating non-hierarchical categories or indexes for the retrieval of data. Folksonomies make it scalable to assign labels to large volumes of data in a collaborative manner and are hence more appropriate for such data than traditional taxonomies established by expert cataloguers~{\cite{HyunwooFrame14}}. At the same time, collaboratively produced folksonomies have several issues, particularly related to incorrect tags and their sparsity~\cite{SunLang11},\cite{MohdSemantic13}. While incorrect tags has been discussed earlier, the sparsity in folksonomy arises as a result of lack of incentive for the users to tag the resources comprehensively and completely. As a result, the online resources are typically associated with low number of tags, preventing effective searching and browsing through the available data~\cite{MohdSemantic13}. 

In order to address the sparsity in folksonomies, several expert systems have been proposed that \textit{recommend} or suggest additional tags for a resource based on the tags already associated with the resource~\cite{SunLang11},\cite{sigurbjornsson2008flickr},\cite{HsiehCollab09},\cite{ChenEstim15}. Most of such works depend on the availability of content-based features such as textual features from documents or blogs~\cite{HsiehCollab09},\cite{ChenEstim15},\cite{SunLang11}, or visual features from images or videos~\cite{ZhaoqiangRegulariz15}, and thus cannot be applied to other domains. 
In addition, extracting and utilizing content based features is known to be computationally expensive and for certain domains, even infeasible \cite{huang2010text}, \cite{song2010taxonomic}, \cite{zanetti2008walk}, \cite{yin2009exploring}, and so the above works may not be applicable to such domains. 
Furthermore, expert systems such as~\cite{MohdSemantic13} utilize purely semantic relationships between tags. While semantic relations as obtained from ontologies such as WordNet~\cite{wordnet} or ODP~\cite{website:ODP} are an important resource for linguistic and machine learning related problems, such relationships fail to capture the information that is characteristic of an available corpus. Consider for instance a corpus of annotated images from Flickr.
% (e.g. publicly available MIR Flickr dataset ~\cite{huiskes08}). Each image is associated with a set of {\em tags} that are applied by users to describe the image. An image typically consists of one or more tags. 
The co-occurrence of tags in a given corpus provides interesting insights into the nature of the data. For example, the 2008 Olympics were held in Beijing and as a result, there exist a large number of images in Flickr having \emph{`2008'} and \emph{`Beijing'} as their tags. Such a relation between \emph{`2008'} and \emph{`Beijing'} cannot be obtained from WordNet or similarly formed hierarchies (such as Open Directory Project, ODP~\cite{website:ODP}) because the semantic relations in the above hierarchies are pre-defined, and do not account for a connection between the two tags. In addition to the above expert systems, works such as~\cite{sigurbjornsson2008flickr} capture tag similarities from a given dataset using tag graphs. Tag graphs usually refer to complete graphs representing pair-wise distances or similarities between tags, which are calculated from a given corpus.  For certain applications, a threshold is applied and only the most important pair-wise connections are retained. However, storing similarities using tag graphs has several issues. Firstly, the pre-defined threshold value that is chosen to construct them can be arbitrary and there is no clear understanding to what its value should be. The pair-wise edges that have their similarity above the defined threshold are the only ones that are retained in the graph and this leads to completely losing of information of those pairs of concepts or tags that have their similarity below the threshold. Depending on the threshold value, the space requirement of tag graphs can vary as $O(N^2)$ where $N$ is the number of concepts or tags in the tag graph, which can be significantly high for large number of tags. In order to keep a handle on the space requirement, a strict threshold value can be chosen which would result in losing pair-wise similarity information for several pairs of concepts or tags. Lastly, depending on the threshold, it is possible that some concepts or tags are disconnected from the rest. This again implies losing relationship information of the concept or tag with others. \cite{sigurbjornsson2008flickr} estimates the number of tags in Flickr in 2008 to be 3.7 million. Storing each similarity value as a floating point occupying 4 bytes would require more than 27 terabytes just to store the pair-wise relationships. 


We attempt to address the above shortcomings in this chapter. We use the term ontological tag tree or simply tag tree to denote undirected weighted tree of concepts (or tags) where the relationships between the concept nodes in the tree are defined only in terms of a scalar weight. 
%Tag trees can approximate the similarities between tags as derived from a given dataset, while having only $O(N)$ space requirement. 
As compared to tag graphs~\cite{sigurbjornsson2008flickr},\cite{liu2009tag}, ontological tag trees are necessarily trees on the set of tags, i.e., are connected and have no simple cycles. We have chosen a spanning tree to represent the relationships between tags because a spanning tree over the set of tags is necessarily connected and does not lead to losing of information due to possibly disconnected components as in tag graphs. Also, the space requirement of a spanning tree is only $O(N)$ for $N$ tags. For 3.7 million tags~\cite{sigurbjornsson2008flickr}, this implies a significant reduction in the space requirement from 27 terabytes ($O(N^2)$) to less than 50 megabytes ($O(N)$). As a result, expert systems can be implemented even on computing devices that do not have a gigantic memory. Ontological tag trees are constructed using the semantic and the data-driven relations between the tags and hence lead to significantly better performance on data-driven tasks than using solely semantic relationships between tags~\cite{MohdSemantic13},\cite{wordnet}. 
%Extracting and utilizing content based features is known to be computationally expensive and for certain domains, even infeasible {\cite{huang2010text}\cite{song2010taxonomic}\cite{zanetti2008walk}\cite{yin2009exploring}}. 
For the constructions of tag trees, we do not utilize content based features, rather we utilize data-driven similarities from tag co-occurrences in the given annotated corpus. As a result, compared to previous expert systems that require extracting and processing content-based (such as visual or textual) features~\cite{HsiehCollab09},\cite{ChenEstim15},\cite{SunLang11},\cite{ZhaoqiangRegulariz15}, tag trees can be used to alleviate sparsity in online folksonomies even in domains where extracting domain specific features may be infeasible or inefficient. This also makes the construction approach not married to a single domain such as annotated text documents/blogs or videos or images.


We illustrate the proposed tag tree construction approach using two large image corpora - one obtained from Flickr, and the other obtained from a set of stock images, with the goal of obtaining a tag tree over the set of tags present in these corpora.
For these corpora, the \emph{co-occurrence count} for a pair of tags is defined as the number of images with which both tags are associated. The normalized co-occurrence counts are a measure of how related two tags are. We assume that the concepts or nodes of the tag tree are the tags, and that the tree construction task is to infer the relations between the tags. The task thus becomes a graph learning problem where the nodes of the graph are the tags, and the relations between tags are represented by undirected edges and their weights in the graph. Unlike the relationships given in ontologies, we do not attempt to give semantic interpretations to the relations between tags. To solve the graph learning problem, we formulate an optimization problem on the space of spanning trees of a suitably constructed similarity graph that is based on semantic relations between tags, as obtained from WordNet, and on the normalized co-occurrence counts of the corpus. We solve the optimization problem using the ``local search'' paradigm by constructing a simple edge exchange based neighborhood on the space of candidate trees. To make the optimization efficient, we initialize our approach using a preliminary tag tree built purely based on semantics from the WordNet hierarchy. The proposed local search based approach is then used to refine the preliminary tag tree based on the corpus statistics.


The evaluation of structures capturing the relationships between different tags or concepts is a difficult task. In the domain of ontologies, there are often no clear quantitative metrics to compare different ontologies that can be built for the same corpus of data. Certain works compare constructed ontologies to a predefined gold standard ontology~\cite{porzel2004task} which is  constructed manually. 
Tag graphs are usually not evaluated explicitly, rather are used in various applications such as Tag Ranking~\cite{liu2009tag}. Since manual evaluations are subjective and are not scalable, in this work, we also propose a novel fully automatic framework to evaluate ontological tag trees over tags using the tag prediction accuracy, given an incomplete set of tags for a resource. 
Furthermore, we also demonstrate that the constructed tag trees can be used to efficiently assign tags to resources in domains where content-based features can be derived. Thus as a second evaluation paradigm, we utilize efficiency: for a given resource with no tags, efficiently predict all the applicable tags.  \\
\indent To summarize, the key contributions of this chapter are as follows:
\begin{enumerate}
	\item We propose a framework to construct an ontological tag tree over tags in a given corpus. \hl{The proposed approach requires constructing a preliminary tag tree using semantics obtained from the WordNet hierarchy.} This preliminary tree is then refined to incorporate data specific relations by performing a novel local search operating on local neighborhoods in the space of spanning trees of a defined similarity graph over the tags. 
	\item \hl{We propose a completely automated framework for evaluating ontological tag trees over tags by posing two data-driven tasks. The first task is defined such that it is does not require content based features to be extracted from resources, in order to assign tags to them. It can thus be applied even to corpora where deriving features from resources is either infeasible or ineffective. The second task is applicable to corpora where domain specific features can be extracted from the resources and classifiers or concept detectors can be trained that can map the content based features of the resource to concepts or tags. 
}
    \item We evaluate the constructed tag trees for two large image corpora using the above evaluation framework and show that it outperforms tag trees built using manually created semantic hierarchies such as WordNet, and commonly used approaches using tag graphs of comparable space requirements, in both prediction accuracy and efficiency. We also demonstrate that by using the constructed tag trees, we can achieve a performance that is very close to or better than that of other techniques, while also achieving several orders reduction in the space requirement. 
\end{enumerate}












\comment{


A commonly used strategy to organize a collection of data is to group it into categories and specify the relationships among the various categories. Ontologies~\cite{fensel2001ontologies}  are often employed to specify predefined relations between categories.

%%%%%%For annotated multimedia content, the association of an image with a tag can be considered as its affiliation with a category. 
%%%%%%Figure ? shows sample images from Flickr and the corresponding tags. Based on the co-occurrence of `park' and `nature' in a large number of image, and the co-occurrence of the correponding semantic concepts in real world, we know that those two tags are related. 
%%%%%In this chapter, we use the term ontology to denote (undirected) weighted graphs of concepts without any specification of the {\em semantic relations} between the concept nodes in the graph.
Conventionally, constructing an ontology~\cite{gruber1995toward} requires significant manual effort. The concepts or categories of the ontology have to be specified, and the relations between the categories defined, all manually. Furthermore, the ontology has to be updated when data belonging to hitherto unseen categories becomes available. Once the ontology has been specified, data samples must be annotated, again manually, to assign them to one or more categories in the ontology so that rules or classifiers can be learned for that category. Therefore, manual techniques for ontological or taxonomic organization of data become especially challenging and cumbersome when there are large amounts of data. Also, ontologies built for one setting are rarely reusable even in other closely related domains. This necessitates the building of an ontology afresh for each new setting. As data could be from one of an ever increasing pool of knowledge domains, manually constructing ontologies for data in each domain is infeasible.  Furthermore, when the data obtained is noisy, as is the case for user generated content on the Internet, the problem is accentuated as more manual effort might be needed to clean up the data followed by ontological organization. 

The challenges associated with the manual construction of ontologies has led to efforts that use semi-automatic~\cite{jaimes2003semi} and fully automatic techniques~\cite{buitelaar2005ontology} in domains such as multimedia and text based ontologies respectively. 
Most existing automatic approaches to ontology construction use text mining techniques to identify the concepts and then define relations between the concepts based on their semantic similarity as obtained from lexical databases such as WordNet~\cite{wordnet}.
%These relations include is-a, is-a-part-of relationships. \\
The semantic relations as obtained from ontologies such as WordNet are an important resource for linguistic and machine learning related problems. However such relationships fail to capture the information that is characteristic of an available corpus. Consider for instance a corpus of annotated images from Flickr (e.g. publicly available MIR Flickr dataset ~\cite{huiskes08}). Each image is associated with a set of {\em tags} that are applied by users to describe the image. 
\comment{ Typical images and their associated tags from this corpus are shown in Fig. ~\ref{fig:flickr}. }
An image typically consists of one or more tags. The co-occurrence of tags in a given corpus provides interesting insights into the nature of the data. For example, the 2008 Olympics were held in Beijing and as a result, there exist a large number of images in Flickr having \emph{`2008'} and \emph{`Beijing'} as their tags. Such a relation between \emph{`2008'} and \emph{`Beijing'} cannot be obtained from WordNet or similarly formed hierarchies (such as Open Directory Project, ODP~\cite{website:ODP}) because the semantic relations in the above hierarchies are pre-defined, and do not account for a connection between the two tags. In order to address this gap between information available in a corpus of data, and manually constructed semantic ontologies, one also needs to account for data specific interactions between the concepts that may not be inferred from prior domain knowledge.

\hl{Another line of works such as~{\cite{feiliang2012demo}}{\cite{ren2012cheap}}{\cite{heymann2006collaborative}}  approach automated construction of ontologies as a two step process comprising of (a) construction of an undirected ontology graph, and (b) derivation of a directed ontology from the ontology graph. The construction of undirected ontology graph is same as the construction of {$tag~graphs$} which are used across several applications involving annotated multimedia content, such as{~\cite{liu2009tag}}. 
Tag graphs usually refer to complete graphs representing pair-wise distances or similarities between tags, which are calculated from a given corpus.  For certain applications, a threshold is applied and only the most important pair-wise connections are retained. Similarly, in~{\cite{feiliang2012demo}}{\cite{ren2012cheap}}{\cite{heymann2006collaborative}}, a threshold is chosen to discard certain edges from the ontology graph. However, the construction of ontology graphs as tag graphs has several issues. Firstly, the pre-defined threshold value that is chosen to construct ontology graphs can be arbitrary and there is no clear understanding to what its value should be. The pair-wise edges that have their similarity above the defined threshold are the only ones that are retained in the graph and this leads to completely losing of information of those pairs of concepts or tags that have their similarity below the threshold. Depending on the threshold value, the space requirement of tag graphs can vary as $O(N^2)$ where $N$ is the number of concepts or tags in the tag graph. In order to keep a handle on the space requirement, a strict threshold value can be chosen which would result in losing pair-wise similarity information for several pairs of concepts or tags. Lastly, depending on the threshold, it is possible that some concepts or tags are disconnected from the rest. This again implies losing relationship information of the concept or tag with others. We attempt to address the above shortcomings of tag graphs in this chapter. We use the term ontological tag tree or simply tag tree to denote undirected weighted tree of concepts where the relationships between the concept nodes in the tree are defined only in terms of a scalar weight i.e. explicit relationship types are not specified.} Once the connections between concepts are determined through an ontological tag tree, their relationship types can be understood from other sources such as ODP, Wikipedia~\cite{website:Wikipedia} or WordNet~\cite{wordnet}, or from corpus statistics. The focus of this chapter is on determining these connections between image tags, and towards this, we propose a novel automatic hybrid approach to building ontological tag trees. 
%We propose a novel  automatic hybrid approach to building an ontological tag tree over the set of image tags.  
As compared to tag graphs, ontological tag trees are necessarily trees on the set of tags, i.e., are connected and have no simple cycles. \hl{We have chosen a spanning tree to represent the relationships between tags because a spanning tree over the set of tags is necessarily connected and does not lead to losing of information due to possibly disconnected components as in tag graphs. Also, the space requirement of a spanning tree is only $O(N)$ for $N$ tags.} Ontological tag trees are constructed using semantic and data-driven relations between the tags. 
Extracting and utilizing content based features is known to be computationally expensive and for certain domains, even infeasible {\cite{huang2010text}\cite{song2010taxonomic}\cite{zanetti2008walk}\cite{yin2009exploring}}. As a result, for the constructions of tag trees, we do not utilize content based features, rather we utilize data-driven similarities from tag co-occurrences in the given annotated corpus. This also makes the constructions approach not married to a single domain such as annotated videos or images.
We illustrate the proposed tag tree construction approach using two large image corpora - one obtained from Flickr, and the other obtained from a set of stock images, with the goal of obtaining a tag tree over the set of tags present in these corpora.
%{There are X images and Y tags in the corpus.}
For these corpora, the \emph{co-occurrence count} for a pair of tags is defined as the number of images with which both tags are associated. The normalized co-occurrence counts are a measure of how related two tags are. We assume that the concepts or nodes of the tag tree are the tags, and that the tree construction task is to infer the relations between the tags. The task thus becomes a graph learning problem where the nodes of the graph are the tags, and the relations between tags are represented by undirected edges and their weights in the graph. Unlike the relationships given in ontologies, we do not attempt to give semantic interpretations to the relations between tags. 
To solve the graph learning problem, we formulate an optimization problem on the space of spanning trees of a suitably constructed similarity graph that is based on semantic relations between tags, as obtained from WordNet, and on the normalized co-occurrence counts of the corpus. We solve the optimization problem using the ``local search'' paradigm by constructing a simple edge exchange based neighborhood on the space of candidate trees. We initialize our approach using a preliminary tag tree built purely based on semantics from the WordNet hierarchy. The proposed local search based approach is then used to refine the preliminary tag tree based on the corpus statistics.
%We solve it using the ``local search'' paradigm by constructing a simple edge exchange based neighbourhood on the space of candidate trees. In order to initialize our approach, we first build a preliminary tag tree using the WordNet hierarchy. Our local search based approach can thus be viewed as data-driven refinement of the preliminary tag tree constructed from semantic ontologies. 
%formulate a local search based approach to refine the graph obtained using WordNet to find a graph that minimizes a novel distance metric defined over the graph.

The evaluation of structures capturing the relationships between different concepts is a difficult task. In the domain of ontologies, there are often no clear quantitative metrics to compare different ontologies that can be built for the same corpus of data. \hl{Most previous approaches for evaluation are $qualitative$ and use manual assessment or expert judgment to evaluate ontologies~{\cite{buitelaar2005ontology}}{\cite{feiliang2012demo}}{\cite{ren2012cheap}}}. Some {\em quantitative} approaches have been proposed to evaluate an ontology by comparing the performance for a specific task to that of a predefined gold standard ontology~\cite{porzel2004task} which is again constructed manually. Both approaches therefore require some manual intervention. Tag graphs are usually not evaluated explicitly, rather are used in various applications such as Tag Ranking~\cite{liu2009tag}. \hl{In this work, we also propose a novel fully automatic framework to evaluate ontological tag trees over tags using two paradigms a) prediction accuracy: given an incomplete set of tags for a resource, infer the rest of the tags, and b) efficiency: for a given resource with no tags, efficiently predict all the applicable tags. } \\
\indent The key contributions of this chapter are as follows:
\begin{enumerate}
	\item We propose a framework to construct an ontological tag tree over tags in a given corpus. \hl{The proposed approach requires constructing a preliminary tag tree using semantics obtained from the WordNet hierarchy.} This preliminary tree is then refined to incorporate data specific relations by performing a novel local search operating on local neighborhoods in the space of spanning trees of a defined similarity graph over the tags. 
	\item \hl{We propose a completely automated framework for evaluating ontological tag trees over tags by posing two data-driven tasks. The first task is defined such that it is does not require content based features to be extracted from resources, in order to assign tags to them. It can thus be applied even to corpora where deriving features from resources is either infeasible or ineffective. The second task is applicable to corpora where domain specific features can be extracted from the resources and classifiers or concept detectors can be trained that can map the content based features of the resource to concepts or tags. 
}
    \item We evaluate the constructed tag trees for two large image corpora using the above evaluation framework and show that it outperforms tag trees built using manually created semantic hierarchies such as WordNet, and commonly used approaches using tag graphs of comparable space requirements, in both prediction accuracy and efficiency. We also demonstrate that by using the constructed tag trees, we can achieve a performance that is very close to that of other techniques with several orders higher space requirement. 
\end{enumerate}

}


\indent Fig.~\ref{fig:todo} illustrates a couple of examples for which the proposed approach captures the inter-dependencies between tags in a qualitatively better form than the tree obtained using WordNet alone. Details on how these trees are obtained are provided in Section~\ref{sec:ConstructionTree}. We first start with a brief discussion on the related literature.
\begin{figure}
\centering
(a)
\epsfig{width=4cm,figure=TagTree/figures/holiday-travel.eps}
\hspace{0.05cm}
\vline
\hspace{0.05cm}
\epsfig{width=4cm,figure=TagTree/figures/holiday-travel2.eps}
\vspace{0.25cm}
\\
(b)
\epsfig{width=5.6cm,figure=TagTree/figures/party-night.eps}
\hspace{0.01cm}
\vline
\hspace{0.05cm}
\epsfig{width=5.6cm,figure=TagTree/figures/party-night2.eps}
\caption{Two examples of subgraphs built using (left) the proposed data-driven approach and (right) corresponding sub-graphs obtained using WordNet. In example (a), `holiday' and `travel' are directly connected using our approach but are separated by multiple hops in the WordNet hierarchy. In example (b), the proposed approach is able to identify `party' as the central node that connects several other party-related tags. For the proposed approach, objective WAH~(\ref{eq:ObjFnWeightedHops}) is utilized as described in Section~\ref{sec:ConstructionTree}.}
\label{fig:todo}
\end{figure}