%% section on related work
\subsection{Related Work}
\label{sub_sec:relatedWork}
%Learning semantics within folksonomies from collaborative tagging systems has been an active research area. This has been motivated by a need to improve user experience of today's information systems, and by complexities arising due to growing available data.

%Several solutions have been proposed to alleviate above issues, tag clustering being one of them. Real world systems such as \textit{Flickr clusters} \cite{FlickrClusters} and studies such as \cite{Begelman06automatedtag}  show that tag clustering is helpful as a means to allow users to explore the information space. However such techniques can only provide grouped entities, instead of modeling data semantics in a structural form. \\ 

In order to discuss the related literature, we study the prior works in terms of works on ontology building, deriving tag relationships, tag recommendation and efficient resource classification, and works on local search paradigm. While we have focused on providing a brief summary of works in these areas that are relevant to our paper, some works may belong to multiple areas. \\ 

\subsubsection{Ontology building}
A commonly used strategy to organize a collection of data is to group it into categories and specify the relationships among the various categories. Ontologies~\cite{fensel2001ontologies}  are often employed to specify predefined relations between categories. Conventionally, constructing an ontology~\cite{gruber1995toward} requires significant manual effort. The concepts or categories of the ontology have to be specified, and the relations between the categories defined, all manually. Furthermore, the ontology has to be updated when data belonging to hitherto unseen categories becomes available. Once the ontology has been specified, data samples must be annotated, again manually, to assign them to one or more categories in the ontology so that rules or classifiers can be learned for that category. Therefore, manual techniques for ontological or taxonomic organization of data become especially challenging and cumbersome when there are large amounts of data. Also, ontologies built for one setting are rarely reusable even in other closely related domains. This necessitates the building of an ontology afresh for each new setting. As data could be from one of an ever increasing pool of knowledge domains, manually constructing ontologies for data in each domain is infeasible.  Furthermore, when the data obtained is noisy, as is the case for user generated content on the Internet, the problem is accentuated as more manual effort might be needed to clean up the data followed by ontological organization. 

The challenges associated with the manual construction of ontologies has led to efforts that use semi-automatic~\cite{jaimes2003semi} and fully automatic techniques~\cite{buitelaar2005ontology} in domains such as multimedia and text based ontologies respectively. 
Most existing automatic approaches to ontology construction use text mining techniques to identify the concepts and then define relations between the concepts based on their semantic similarity as obtained from lexical databases such as WordNet~\cite{wordnet}.

Aside of these, other works on ontology building use some form of clustering to combine similar terms or keywords together to form concepts.  First, a similarity metric is defined between tags, words or concepts, and then a hierarchical clustering algorithm is utilized to form a dendrogram with the concepts as the leaves of the formed tree. The hierarchical clustering algorithm can be either bottom-up (agglomerative) or top-down (divisive). Such a procedure creates auxiliary concepts in the tree representing combinations of multiple concepts of interest. For example~\cite{neshati2007taxonomy} uses hierarchical clustering based on a compound similarity measure between words. The similarity score is obtained by using a neural network model on syntactical information and corpus based similarities. However such techniques can only group related concepts together at different hierarchical levels, instead of modeling the inter-dependencies in the form of a graph on the concepts. 
%provide grouped entities, instead of modeling data semantics in a structural form. 
In \cite{dietz2012taxolearn}, given a corpus corresponding to a domain, the relations between important concepts are learned with the help of WordNet or by using search engine. Hierarchical clustering is employed to construct a domain specific dendrogram as mentioned above.

Works such as \cite{hearst1992automatic}  and \cite{cimiano2005learning} utilize natural language based grammar rules to learn hierarchies between text entities. Semi-automatic techniques for ontology construction such as Text2Onto~\cite{cimiano2005text2onto} assist the user in constructing ontologies from a given set of text based data. Similar techniques have been attempted in the domain of annotated multimedia content, such as images and videos~\cite{jaimes2003semi}. Fully automatic techniques such as OntoLearn etc.~\cite{velardi2005evaluation}, \cite{navigli2003ontology}, \cite{mani2004automatically} use natural language processing and machine learning to extract concepts and relations from data. For a good review of ontology learning from text see ~\cite{buitelaar2005ontology}.  These works cannot be applied outside of the domain of natural language, since they depend at least in part upon grammatical speech. 


\subsubsection{Deriving tag relationships}

The organization of tags or concepts obtained from different domains has also been explored. Tag clustering has been employed in systems such as \textit{Flickr clusters} \cite{FlickrClusters} and studies~\cite{Begelman06automatedtag} show that it is helpful as a means to allow users to explore the information space of tags. For annotated images, \cite{SubsumptionFlickr} proposed the application of a co-occurrence based subsumption model from \cite{sanderson1999deriving}, to learn whether a tag {\em subsumes} another. \cite{griffin2008learning} uses the category confusion matrix to cluster similar categories together in a hierarchical manner. To construct an ontology for a set of tags, \cite{WordnetHierarchyConstruct} maps the tags to WordNet and leverages WordNet's hierarchy. Tag graphs have been utilized for various applications such as tag ranking~\cite{liu2009tag} to represent the pair-wise similarities or distances between tags. While several works use tag graphs as complete graphs on the set of tags, others choose set of edges that have their distance lower than a heuristically chosen threshold~\cite{heymann2006collaborative}. In general, tag graphs have $O(N^2)$ edges with correspondingly large storage requirement for large values of $N$. \hl{For example, {\cite{sigurbjornsson2008flickr}} estimates the number of tags in Flickr in 2008 to be 3.7 million. Storing each similarity value as a floating point occupying 4 bytes would require more than 27 terabytes to store the pair-wise relationships, as compared to under 50 megabytes as required by the proposed tag tree. This eliminates the need to have computing devices with gigantic memory in order to operate on a large number of tags or concepts for tasks such as tag prediction, resource annotation, etc.  \\ 
\indent In the domain of annotated images, there exist works that determine semantic relationships between concepts using visual features {\cite{wu2008flickr}} and using visual features and tags {\cite{katsurai2013cross}}. 
The approaches proposed in these works are specific to the domain of images and require visual-feature based representation of images. As mentioned in {\cite{huang2010text}\cite{song2010taxonomic}\cite{zanetti2008walk}} and {\cite{yin2009exploring}}, extracting and utilizing content based features can be computationally expensive and even infeasible in certain domains. In addition to above, the space requirement of these works varies as $O(N^2)$ where $N$ is the number of tags or concepts. Our work is different from the above works since we propose a tag tree construction approach which is not dependent on the availability of content-based features from the resources and has a space requirement of only $O(N)$. 


\subsubsection{Tag recommendation and efficient resource classification} 

While several expert systems such as~\cite{DeepaFolkso14}, \cite{ShuhuiAuthor15}, \cite{ZhengRecom10}, \cite{MohdSemantic13}, \cite{HyunwooFrame14} address the broader problem of information overload, others focus on addressing the sparsity of online folksonomies through approaches such as tag recommendation and efficient resource classification. In this section we provide a brief summary of the existing literature in the latter category, since such expert systems are closer to our evaluation tasks. 


In our first evaluation task, we attempt to predict certain tags associated with a resource while having visibility to other tags associated with the same resource. There exist works in literature that utilize domain-specific features to associate tags to a resource or to determine the relevance of tags to a given resource. For instance, {\cite{li2009learning}} uses visual similarity to determine neighbors of a test images and then aggregates their tags by voting. {\cite{wu2009distance}} learns a distance metric to determine images that are $close$ to a given image based on the visual content, and then determines tag relevance for the image. \cite{HsiehCollab09} builds a desktop collaborative tagging system to enable collaborative workers to tag their offline documents. \cite{ChenEstim15} approaches tag recommendation as a translation problem to translate the textual description to tags, while \cite{SunLang11} uses language modeling to recommend tags for blogs and documents. These works require extraction of domain-specific, in this case visual and textual features from the resource (images or blogs/documents) and cannot be applied to other domains or corpora where deriving features from resources is either infeasible or ineffective {\cite{huang2010text}\cite{song2010taxonomic}\cite{zanetti2008walk}\cite{yin2009exploring}}.
Aside of these works, works such as~\cite{MohdSemantic13} use semantic similarity between tags as obtained from WordNet, to address information overload. As discussed in Section~\ref{sec:Introduction}, purely semantics based systems fail to capture data-specific characteristics, thus leading to poor performance on data-driven tasks. 
The tag prediction task as proposed in our paper is defined such that it can even be applied to corpora where deriving content based features from resources is either ineffective or infeasible.
The proposed task is somewhat similar to the tag recommendation application in  {\cite{katsurai2013cross}} and in {\cite{sigurbjornsson2008flickr}}. However, the procedure for tag recommendation in {\cite{katsurai2013cross}} and  {\cite{sigurbjornsson2008flickr}} requires manual labeling of tags to measure the performance. Since manual labeling is subjective, irregular and not scalable to large numbers of testing resources, we define a tag prediction task where given an incomplete set of tags associated with a resource, we attempt to predict the rest of the tags. Using such an approach, our evaluation is completely automated and does not require any human assistance or labeling. As mentioned above, {\cite{katsurai2013cross}} requires visual features to obtain the concept similarities. We compare the performance of the constructed tag trees with the symmetric sum based tag recommendation approach as outlined in {\cite{sigurbjornsson2008flickr}} and show that we achieve almost similar performance with several orders reduction in the space requirement.  

\indent In the second data-driven evaluation task, we utilize tag trees to associate tags to resources based on their content. Specifically, for domains where content-based features can be extracted, we show that using the constructed tag trees, it is possible to determine which concept detectors should be tested for the test resource, thereby making the resource classification more efficient. Compared to {\cite{li2013classifying}}, our approach does not require training of faster and efficient classifiers for this task and can utilize pre-trained binary classifiers as concept detectors for different tags. The image annotation application in {\cite{wu2008flickr}} associates a test image with tags based on applying all concept detectors on the test image and using the predicted image likelihood under the Dual Cross-Media Relevance model~{\cite{liu2007dual}}. Compared to {\cite{wu2008flickr}}, our approach does not require applying all concept detectors corresponding to the tags, rather our objective is to determine the concept detectors to apply on a given test image. In order to demonstrate that the proposed approach for the second data-driven task is not applicable to only a single domain, we provide evaluation results based on two types of modalities - visual, and textual. We have used two large sized image corpora to demonstrate the above evaluation tasks. 
%In addition, for the second task, we have used different types of modalities (textual and visual) to show the applicability of the proposed approach to different modalities that can be used to represent the content of a given resource. 
The tasks in Section {\ref{sec:Expts}} are defined such that they can be used to evaluate the constructed tag trees. Most of the works as discussed above do not offer a way to do so. 


\subsubsection{Local search paradigm}

The use of local search methods in combinatorial optimization has a long history~\cite{localBook}. 
The paradigm has been extensively studied~\cite{johnson}\cite{aarts} due to its practical success on many NP hard problems and also for the insights it provides on the structure of discrete optimization problems.
%Owing to its practical success on many NP-Hard problems as well as the insights it provides on the structure of discrete optimization problems, this paradigm has been extensively studied~\cite{johnson}\cite{aarts}.
The use of exchange neighborhoods was introduced by~\cite{croes} and~\cite{lin} for solving the Traveling Salesman Problem and has since been successfully applied to a wide variety of problems. See~\cite{localBook} for a comprehensive survey. 




\indent We formulate an automated approach for building an ontological tag tree with $(N-1)$ edges for $N$ tags using WordNet followed by a data-driven refinement.  
To our knowledge, the formulation of the onntological tag tree construction as an optimization problem on the space of spanning trees and its solution using the ``local search'' paradigm is completely novel. We use a variant of the edge-exchange  method to construct the neighborhood on the solution space. 















\comment{
There have been several works addressing the general topic of ontology building. Most approaches use some form of clustering to combine similar terms or keywords together to form concepts.  First, a similarity metric is defined between tags, words or concepts, and then a hierarchical clustering algorithm is utilized to form a dendrogram with the concepts as the leaves of the formed tree. The hierarchical clustering algorithm can be either bottom-up (agglomerative) or top-down (divisive). Such a procedure creates auxiliary concepts in the tree representing combinations of multiple concepts of interest. For example~\cite{neshati2007taxonomy} uses hierarchical clustering based on a compound similarity measure between words. The similarity score is obtained by using a neural network model on syntactical information and corpus based similarities. However such techniques can only group related concepts together at different hierarchical levels, instead of modeling the inter-dependencies in the form of a graph on the concepts. 
%provide grouped entities, instead of modeling data semantics in a structural form. 
In \cite{dietz2012taxolearn}, given a corpus corresponding to a domain, the relations between important concepts are learned with the help of WordNet or by using search engine. Hierarchical clustering is employed to construct a domain specific dendrogram as mentioned above. \hl{Works such as~{\cite{feiliang2012demo}}{\cite{ren2012cheap}} construct ontologies using a two step process comprising of construction of an ontology graph and derivation of the ontology. Ontology graphs are constructed as tag graphs on the set of tags. The shortcomings of such an approach are discussed in Section{~\ref{sec:Introduction}}. 
}

Works such as \cite{hearst1992automatic}  and \cite{cimiano2005learning} utilize natural language based grammar rules to learn hierarchies between text entities. Semi-automatic techniques for ontology construction such as Text2Onto~\cite{cimiano2005text2onto} assist the user in constructing ontologies from a given set of text based data. Similar techniques have been attempted in the domain of annotated multimedia content, such as images and videos~\cite{jaimes2003semi}. Fully automatic techniques such as OntoLearn etc.~\cite{velardi2005evaluation, navigli2003ontology, mani2004automatically} use natural language processing and machine learning to extract concepts and relations from data. For a good review of ontology learning from text see ~\cite{buitelaar2005ontology}.  These works cannot be applied outside of the domain of natural language, since they depend at least in part upon grammatical speech. 

%The construction of tag ontologies and taxonomies specifically for image corpora such as Flickr has also been explored. 
The organization of tags or concepts obtained from multimedia corpora has also been explored. Tag clustering has been employed in systems such as \textit{Flickr clusters} \cite{FlickrClusters} and studies~\cite{Begelman06automatedtag} show that it is helpful as a means to allow users to explore the information space of tags. For annotated images, \cite{SubsumptionFlickr} proposed the application of a co-occurrence based subsumption model from \cite{sanderson1999deriving}, to learn whether a tag {\em subsumes} another. \cite{griffin2008learning} uses the category confusion matrix to cluster similar categories together in a hierarchical manner. To construct an ontology for a set of tags, \cite{WordnetHierarchyConstruct} maps the tags to WordNet and leverages WordNet's hierarchy. Tag graphs have been utilized for various applications such as tag ranking~\cite{liu2009tag} to represent the pair-wise similarities or distances between tags. While several works use tag graphs as complete graphs on the set of tags, others choose set of edges that have their distance lower than a heuristically chosen threshold~\cite{heymann2006collaborative}. In general, tag graphs have $O(N^2)$ edges with correspondingly large storage requirement for large values of $N$. \hl{For example, {\cite{sigurbjornsson2008flickr}} estimates the number of tags in Flickr in 2008 to be 3.7 million. Storing each similarity value as a floating point occupying 4 bytes would require more than 27 terabytes to store the pair-wise relationships, as compared to under 50 megabytes as required by the proposed tag tree. This eliminates the need to have computing devices with gigantic memory in order to operate on a large number of tags or concepts for tasks such as tag prediction, resource annotation, etc.  \\ 
\indent In the domain of annotated images, there exist works that determine semantic relationships between concepts using visual features {\cite{wu2008flickr}} and using visual features and tags {\cite{katsurai2013cross}}. 
The approaches proposed in these works are specific to the domain of images and require visual-feature based representation of images. As mentioned in {\cite{huang2010text}\cite{song2010taxonomic}\cite{zanetti2008walk}} and {\cite{yin2009exploring}}, extracting and utilizing content based features can be computationally expensive and even infeasible in certain domains. In addition to above, the space requirement of these works varies as $O(N^2)$ where $N$ is the number of tags or concepts. Our work is different from the above works since we propose a tag tree construction approach which is not dependent on the availability of content-based features from the resources and has a space requirement of only $O(N)$. 
%In a way, our work is orthogonal to the above works since the proposed local search based approach could also be used to obtain a tag tree based on the semantic similarity as defined in {\cite{wu2008flickr}} or {\cite{katsurai2013cross}}, and utilize space only of the order of $O(N)$ for storing the pair-wise similarities. 
} \\
\indent In most of the above mentioned works on ontology construction, evaluation of a constructed ontology is either done manually or by comparing it to a gold standard ontology. The latter is obtained either from existing ontologies such as ODP and WordNet or is manually constructed. Manual involvement in evaluation of ontologies makes the process subjective and non-scalable. While several applications pertaining to annotated  multimedia content utilize tag graphs, the latter usually are not evaluated explicitly. \hl{For evaluation of the constructed tag trees, we propose two automated data-driven tasks. In the first task, we attempt to predict certain tags associated with a resource while having visibility to other tags associated with the same resource. There exist works in literature that utilize visual content to associate tags to an image or to determine the relevance of tags to a given image. For instance, {\cite{li2009learning}} uses visual similarity to determine neighbors of a test images and then aggregates their tags by voting. {\cite{wu2009distance}} learns a distance metric to determine images that are $close$ to a given image based on the visual content, and then determines tag relevance for the image. These works require extraction of domain-specific, in this case visual features from the resource (image) and cannot be applied to corpora where deriving features from resources is either infeasible or ineffective {\cite{huang2010text}\cite{song2010taxonomic}\cite{zanetti2008walk}\cite{yin2009exploring}}. The tag prediction task as proposed in our paper is defined such that it can even be applied to corpora where deriving content based features from resources is either ineffective or infeasible.
The proposed task is somewhat similar to the tag recommendation application in  {\cite{katsurai2013cross}} and in {\cite{sigurbjornsson2008flickr}}. However, the procedure for tag recommendation in {\cite{katsurai2013cross}} and  {\cite{sigurbjornsson2008flickr}} requires manual labeling of tags to measure the performance. Since manual labeling is subjective, irregular and not scalable to large numbers of testing resources, we define a tag prediction task where given an incomplete set of tags associated with a resource, we attempt to predict the rest of the tags. Using such an approach, our evaluation is completely automated and does not require any human assistance or labeling. As mentioned above, {\cite{katsurai2013cross}} requires visual features to obtain the concept similarities. We compare the performance of the constructed tag trees with the symmetric sum based tag recommendation approach as outlined in {\cite{sigurbjornsson2008flickr}} and show that we achieve almost similar performance with several orders reduction in the space requirement.  \\
\indent In the second data-driven evaluation task, we utilize tag trees to associate tags to resources based on their content. Specifically, for a given test image, we utilize the visual content to determine which concept detectors should be tested for the test image. Compared to {\cite{li2013classifying}}, our approach does not require training of faster and efficient classifiers for this task and can utilize pre-trained binary classifiers as concept detectors for different tags. The image annotation application in {\cite{wu2008flickr}} associates a test image with tags based on applying all concept detectors on the test image and using the predicted image likelihood under the Dual Cross-Media Relevance model~{\cite{liu2007dual}}. Compared to {\cite{wu2008flickr}}, our approach does not require applying all concept detectors corresponding to the tags, rather our objective is to determine the concept detectors to apply on a given test image. The proposed approach can be applied to corpora where it is feasible to extract content-based features from the resources. We have used two large sized image corpora to demonstrate the above evaluation tasks. In addition, for the second task, we have used different types of modalities (textual and visual) to show the applicability of the proposed approach to different modalities that can be used to represent the content of a given resource. The tasks in Section {\ref{sec:Expts}} are defined such that they can be used to evaluate the constructed tag trees. Most of the works as discussed above do not offer a way to do so. \\
}
%capture semantic relations between concepts that are determined to be important in a corpus, they fail to encode the information pertaining to interaction of these concepts. A comment in line with our hypothesis has been mentioned in \cite{SubsumptionFlickr} - matching concepts to existing ontologies such as WordNet may be inherently less noisy, but since WordNet is based upon standard English vocabulary, it may be difficult to adapt such models to the vocabulary that emerges in tagging applications. 
%Most existing approaches for ontological tag tree construction are in the domain of textual data and are semiautomatic requiring some manual intervention to build the ontology. 
%A probabilistic model, termed the \emph{subsumption model} for inferring the relations between tags has also be been proposed for the task of tag taxonomy learning for a corpus of images from Flickr ~\cite{SubsumptionFlickr}.% Difference with clustering based techniques where each tag is leaf node (how do we compare against them?) (Several papers can be cited here) \\
%- Using learnt hierarchy for classification \cite{song2010taxonomic} \cite{dumais2000hierarchical} \cite{griffin2008learning} \\
\indent We formulate an automated approach for building an ontological tag tree with $O(N)$ edges for $N$ tags using WordNet followed by a data-driven refinement.  
To our knowledge, the formulation of the onntological tag tree construction as an optimization problem on the space of spanning trees and its solution using the ``local search'' paradigm is completely novel. We use a variant of the edge-exchange  method to construct the neighborhood on the solution space. 

The use of local search methods in combinatorial optimization has a long history~\cite{localBook}. 
The paradigm has been extensively studied~\cite{johnson}\cite{aarts} due to its practical success on many NP hard problems and also for the insights it provides on the structure of discrete optimization problems.
%Owing to its practical success on many NP-Hard problems as well as the insights it provides on the structure of discrete optimization problems, this paradigm has been extensively studied~\cite{johnson}\cite{aarts}.
The use of exchange neighborhoods was introduced by~\cite{croes} and~\cite{lin} for solving the Traveling Salesman Problem and has since been successfully applied to a wide variety of problems. See~\cite{localBook} for a comprehensive survey. 
}

\cite{WWW14poster} presents the preliminary results using the proposed approach. We next discuss the problem statement addressed in this paper, followed by the proposed tree construction approach. 
