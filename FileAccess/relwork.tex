\section{Related Work}
\label{sec:relatedwork}
%%
Several previous approaches have addressed the topic of file access
prediction, but with the goal of improving the I/O performance of
storage systems~\cite{Amer02fileaccess, Xia08farmer, kroeger01-usenix,
  yeh02-hpcs, yeh01-mascots, yeh01-ispass, Whittle03usingmultiple,
  Paris-stochastic}.  They aim at reducing the widening gap between
CPU and disk storage performance by prefetching files to the cache
memory.
%%
%%  FIXME: The below examples are not serving any purpose.
%%
\comment {
For instance, Amer et al.\cite{Amer02fileaccess} proposes
predictors that can reduce the number of predictions made in order to
increase the likelihood of making correct
predictions. \cite{Xia08farmer} proposes a model to infer file
correlations to optimize performance of peta-scale storage systems
while \cite{Whittle03usingmultiple} proposes a composite file access
predictor that uses multiple heuristics to predict likely successors
of files.
}
%%
%%
%%
There are a couple of key differences between our approach and the
previous ones.  Such approaches focused on modeling accesses
patterns that are generated by automated activities, whereas our focus
is on activities that are manually initiated by physical users.  This
allows us to consider certain features that could be prohibitively
expensive for caching applications.  None of the previous approaches
make predictions for completely new content.  On the other hand, by
leveraging innovative features derived from metadata and user
collaborations, our approach can make fairly good predictions for new
content.  In fact, our goal is to be able to discover useful new
content, which could be present in new or modified files.

The approach from Song et al.\cite{song14-tis} is closest to ours in
terms of providing recommendations to knowledge workers.  This
approach infers abstract tasks and frequently used workflow patterns
from historical user activity.  It then makes recommendations based on
the workflows that match current activities of the user.  Although
this approach extends beyond simple file matching, it cannot make
predictions for new content.

In terms of features, our approach is significantly different from any
of the previous approaches.  Our features are derived from rich file
metadata, including file name, path, extension, and file system
hierarchy.  These features allow our system to compare new file
activities to the learned activities in the past in terms of
similarity in the metadata attributes.  More importantly, our system
leverages metadata features to automatically generate features for
collaborative filtering in an interesting way so as to address the
cold start problem (i.e., inability to provide recommendation for new
content).  In that respect, our approach is better than the
traditional collaborative filtering approaches~\cite{linden2003amazon,
  breese1998empirical} because they suffer from the cold-start
problem.

Lastly, advanced machine learning models, e.g., factorization
machines~\cite{rendle2010factorization}, topic
models~\cite{nagori11-etncc, Ovsjanikov_topicmodeling}, and deep
neural networks~\cite{salakhutdinov2007restricted,hinton2006fast},
could be used to model access patterns. However, these techniques are
complementary and can be applied to make our system more effective.
Nonetheless, we demonstrate that simple linear SVM model can be
effectively used to build a practical system.  Most importantly, it
allows us to develop the optimization technique AFMS for significantly
reducing the classification time and improving the scalability of our
system.  This technique, in addition to the basic modeling approach,
constitutes a novel contribution of our research.

\comment{
%%% COMMENTED -- OLD CONTENT. 
% removed Ellard03attribute-basedprediction (it predicts only
% read/write pattern for newly created files

Prior works on modeling file access patterns have been mostly focused
on performance enhancement of storage systems, e.g., reducing I/O
latency by prefetching~\cite{Amer02fileaccess, Xia08farmer,
  kroeger01-usenix, yeh02-hpcs, yeh01-mascots, yeh01-ispass,
  Whittle03usingmultiple, Paris-stochastic}.  These systems make
predictions for only existing files, whereas our approach can make
predictions for newly created files too.  However, our focus is on
recommendation rather than caching.
%%  file caching don't do NLP, tokenization, etc.

The approach by Song et al.~\cite{song14-tis} is closer to our work
since it aims at assisting knowledge workers by recommending files and
actions.  It uses a data mining technique to first group similar files
into abstract tasks, and then mines frequent sequences of abstract
tasks into workflows. It then makes recommendations by identifying the
workflow that best matches the current file usage pattern of the
user. While the technique attempts to generalize beyond exact file
matching, it cannot provide recommendations for new files. In
contrast, we train personalized machine learning models that provide
recommendations even for new files. For evaluation of our approach, we
use only those test files that are new with respect to the
training files.
%As mentioned in Section~\ref{sec:introduction}, assisting knowledge workers to discover new content is essential especially given the high rate of new content generation~\cite{IDCDataGrowth}.
The ability to recommend new content is important in order to connect
knowledge workers with new data which is being generated at a
tremendous rate~\cite{IDCDataGrowth}.

Unlike all the previous works, we use much richer file metadata
including filename, path, file system hierarchy, extensions and
collaborative filtering in our models. As a result of this, our work
can also supplement existing Data Governance systems with predictive
capabilities. While our approach is not ideal for file caching in
performance sensitive applications, it could be effective in cloud
services to reduce network latency by caching files on client-facing
web servers or directly on clients.  It could also be useful for
scenarios with intermittent connectivity, such as choosing files to
cache on mobile devices.
%In contrast, our metadata features are broader, capturing not only file similarity but also hierarchical folder structure that adds to the quality of our classification model.
%% FIXME: add a referencd here..

The personalized model based file recommendation as proposed in our
paper is a content-based recommendation system. As compared to
traditional collaborative filtering based recommender
systems~\cite{linden2003amazon, breese1998empirical}, our approach
does not suffer from cold start problem, i.e., inability to recommend
a new item (file). It should however be noted that unlike traditional
collaborative filtering techniques, we do not use actual access
information. Rather, we predict the access likelihood of a user for a
particular test item and combine it with metadata features. This
enables us to circumvent the cold start problem, and thus benefit from
collaborative filtering.

Finally, advanced machine learning models such as factorization
machines~\cite{rendle2010factorization}, deep neural
networks~\cite{salakhutdinov2007restricted,hinton2006fast} and topic
models~\cite{nagori11-etncc, Ovsjanikov_topicmodeling} can also be
employed for modeling file access predictions.
%are often used in recommendation systems.  
To a large extent, these techniques are complimentary and can
contribute in making our models more effective.  Notwithstanding, we
approach the problem as a classification problem and show reasonable
effectiveness even with a simple Linear SVM-based model.  Our focus is
more on the domain specific application, with the goal of extracting
meaningful features from file metadata and user activities.

%% Several works have addressed the topic of file access prediction with
%% a goal of improving the I/O performance of storage
%% systems~\cite{Amer02fileaccess, Xia08farmer, kroeger01-usenix,
%%   yeh02-hpcs, yeh01-mascots, yeh01-ispass, Whittle03usingmultiple,
%%   Paris-stochastic}. These works attempt to reduce the widening gap
%% between CPU and disk storage performance by prefetching files to
%% cache. For instance, \cite{Amer02fileaccess} proposes predictors that
%% can reduce the number of predictions made in order to increase the
%% likelihood of making correct predictions. \cite{Xia08farmer} proposes
%% a model to infer file correlations to optimize performance of
%% peta-scale storage systems while \cite{Whittle03usingmultiple}
%% proposes a composite file access predictor that uses multiple
%% heuristics to predict likely successors of files.  While these works
%% address a relevant problem to ours, there are three key differences
%% with our work. These works focus on modeling regular file accesses
%% patterns that are primarily made by computer programs. In comparison,
%% our work is concerned with file access predictions for user activity
%% that is not from computer programs. The second key difference is that
%% the above works do not have the ability to make predictions for
%% content that is completely new while our work overcomes this shortcoming by constructing and
%% utilizing generalizable models that capture access patterns in terms
%% of metadata and collaboration based features. In fact, given the high rate of content generation in enterprises, our system is focused on new files and our evaluation framework is designed such that the test data is comprised of files that are new with respect to training. 
%% Lastly, our work is focused on recommendation applications
%% and not caching.

%% \cite{song14-tis} is closer to our work since it focuses on assisting
%% knowledge workers by recommending relevant content and
%% actions. \cite{song14-tis} infers abstract tasks and frequently used
%% workflow patterns from historical user activity. Recommendations to
%% users are made based on the workflows that their activities correlate
%% with. The approach proposed in \cite{song14-tis} is more generalizable
%% than simple file matching and can handle differently ordered
%% elements. Their approach, however, can not be extended to recommend
%% new content. Given the tremendous rate of data
%% generation~\cite{IDCDataGrowth} and of file edits\footnote{as noted in
%%   Section~\ref{sec:scalability}}, our work constructs personalized
%% models that are generalizable and can recommend, and help knowledge
%% workers discover, content that is completely new.

%% Unlike previous systems, we utilize features constructed
%% based on rich file metadata including file name, path, extensions,
%% user collaboration, file system hierarchy and folder information. As a
%% result, our system can also be used to assign likelihoods to user activities
%% based on how similar they are to the \textit{expected} activities. Our
%% work can thus complement Data Governance systems by providing
%% predictive capabilities.

%% The system proposed in this paper is primarily a \textit{content based
%%   recommendation system} which is supplemented by features based on
%% user collaboration. Compared to our modeling approach, traditional
%% approaches for Collaborative Filtering~\cite{linden2003amazon,
%%   breese1998empirical} suffer from the cold-start problem, i.e., inability
%% to provide recommendations for new
%% content. Section~\ref{sec:CFdetails} discusses how our system avoids
%% the cold start problem and can still leverage user collaboration
%% to improve its predictions.

%% Lastly, advanced machine learning models for example factorization
%% machines~\cite{rendle2010factorization}, topic
%% models~\cite{nagori11-etncc, Ovsjanikov_topicmodeling} and deep neural
%% networks~\cite{salakhutdinov2007restricted,hinton2006fast} have been
%% developed that could be used to model access patterns. These
%% techniques are complementary to our work and could be used to make the
%% proposed system more effective and efficient. Using a simple linear SVM to
%% model access patterns, we have demonstrated that it is feasible to recommend new content by using metadata
%% and collaboration based features. We have also proposed a novel approach that can potentially be extended to other content based recommender systems to significantly reduce their testing computational requirement by selecting the models to be applied on each test instance. 
%% Our work can be extended in several
%% directions, as discussed next.  \comment{
%% %%% COMMENTED -- OLD CONTENT. 
%% % removed Ellard03attribute-basedprediction (it predicts only
%% % read/write pattern for newly created files

%% Prior works on modeling file access patterns have been mostly focused
%% on performance enhancement of storage systems, e.g., reducing I/O
%% latency by prefetching~\cite{Amer02fileaccess, Xia08farmer,
%%   kroeger01-usenix, yeh02-hpcs, yeh01-mascots, yeh01-ispass,
%%   Whittle03usingmultiple, Paris-stochastic}.  These systems make
%% predictions for only existing files, whereas our approach can make
%% predictions for newly created files too.  However, our focus is on
%% recommendation rather than caching.
%% %%  file caching don't do NLP, tokenization, etc.

%% The approach by Song et al.~\cite{song14-tis} is closer to our work
%% since it aims at assisting knowledge workers by recommending files and
%% actions.  It uses a data mining technique to first group similar files
%% into abstract tasks, and then mines frequent sequences of abstract
%% tasks into workflows. It then makes recommendations by identifying the
%% workflow that best matches the current file usage pattern of the
%% user. While the technique attempts to generalize beyond exact file
%% matching, it cannot provide recommendations for new files. In
%% contrast, we train personalized machine learning models that provide
%% recommendations even for new files. For evaluation of our approach, we
%% use only those test files that are new with respect to the
%% training files.
%% %As mentioned in Section~\ref{sec:introduction}, assisting knowledge workers to discover new content is essential especially given the high rate of new content generation~\cite{IDCDataGrowth}.
%% The ability to recommend new content is important in order to connect
%% knowledge workers with new data which is being generated at a
%% tremendous rate~\cite{IDCDataGrowth}.

%% Unlike all the previous works, we use much richer file metadata
%% including filename, path, file system hierarchy, extensions and
%% collaborative filtering in our models. As a result of this, our work
%% can also supplement existing Data Governance systems with predictive
%% capabilities. While our approach is not ideal for file caching in
%% performance sensitive applications, it could be effective in cloud
%% services to reduce network latency by caching files on client-facing
%% web servers or directly on clients.  It could also be useful for
%% scenarios with intermittent connectivity, such as choosing files to
%% cache on mobile devices.
%% %In contrast, our metadata features are broader, capturing not only file similarity but also hierarchical folder structure that adds to the quality of our classification model.
%% %% FIXME: add a referencd here..

%% The personalized model based file recommendation as proposed in our
%% paper is a content-based recommendation system. As compared to
%% traditional collaborative filtering based recommender
%% systems~\cite{linden2003amazon, breese1998empirical}, our approach
%% does not suffer from cold start problem, i.e., inability to recommend
%% a new item (file). It should however be noted that unlike traditional
%% collaborative filtering techniques, we do not use actual access
%% information. Rather, we predict the access likelihood of a user for a
%% particular test item and combine it with metadata features. This
%% enables us to circumvent the cold start problem, and thus benefit from
%% collaborative filtering.

%% Finally, advanced machine learning models such as factorization
%% machines~\cite{rendle2010factorization}, deep neural
%% networks~\cite{salakhutdinov2007restricted,hinton2006fast} and topic
%% models~\cite{nagori11-etncc, Ovsjanikov_topicmodeling} can also be
%% employed for modeling file access predictions.
%% %are often used in recommendation systems.  
%% To a large extent, these techniques are complimentary and can
%% contribute in making our models more effective.  Notwithstanding, we
%% approach the problem as a classification problem and show reasonable
%% effectiveness even with a simple Linear SVM-based model.  Our focus is
%% more on the domain specific application, with the goal of extracting
%% meaningful features from file metadata and user activities.


\comment { FIXME:

Notes:
- focus on application rather than techniques

- comparison with file prefetching:  they don't predict new files

- comparison with netflix-style collaborative filtering: the dataset is
  different: no folder-like features; moreover, these techniques are
  complimentary.

- comparison with advance techniques(e.g., topic modeling) : these are complimentary techniques.

}

\comment {

Attribute-based prediction of file properties

FARMER: a novel approach to file access correlation mining and evaluation reference model for optimizing peta-scale file system performance

Design and Implementation of a Predictive File Prefetching Algorithm

Performing file prediction with a program-based successor model

Using multiple predictors to improve the accuracy of file access predictions

Using program and user information to improve file prediction performance.

Increasing predictive accuracy by prefetching multiple program and user specific files

A Program-and-User Based File Access Prediction Model

Reducing File System Latency Using a Predictive Approach

The Case for Efficient File Access Pattern Modeling

Context-Aware Prefetching at the Storage Server

File Access Prediction Using Neural Networks

File access prediction with adjustable accuracy

A stochastic approach to file access prediction

The methods of data prefetching based on user model in cloud computing

Topic Modeling for Personalized Recommendation of Volatile Items

LDA Based Integrated Document Recommendation Model for e-Learning Systems 

}
}
