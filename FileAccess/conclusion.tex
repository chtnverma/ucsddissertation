\section{Conclusion}
\label{sec:conclusion}

This paper presents a system to assist knowledge workers in
discovering useful new or modified content.  The system builds
personalized user models using features derived from file metadata and
user collaboration.  Our experiments showed that the basic modeling
approach does not scale well under heavy workloads. To address this
problem, we developed a novel optimization technique that improves
runtime performance without sacrificing prediction accuracy.  We also
showed how the technique can be adapted to improve the performance
significantly at marginal cost of the predictive correctness.

For future work, we could improve our system along different
directions.  Leveraging interactions observed between the features, we
could reduce dimensionality of the data for better efficiency and
effectiveness.  We could implement a weighing scheme that gives more
importance to the recent activity as it is more reflective of the
current preferences of users.  Finally, we could incorporate features
that capture producer-consumer relationship between users.

%% This paper presented a system that can assist knowledge workers to cope up with the growing enterprise data by recommending relevant new or modified content. The system trains file metadata models that can also leverage user collaboration to make more accurate predictions. 
%% %This paper presents a file recommender system that recommends relevant new or modified content to knowledge workers to assist them to cope up with the tremendous growth rate of enterprise data. The system uses file metadata and past user activity to train personalized user models and it is shown that collaboration demonstrated by the users can also be leveraged to make models more accurate. 
%% While experiments based on real enterprise activity demonstrated the feasibility of using the proposed system to recommend new content, it was observed that high testing computational requirement poses challenges for its deployment in enterprises, especially on large and active shares. In order to address this shortcoming, we proposed a novel model selection approach AFMS which is designed to leverage properties of the machine learning models used in our system, and of the enterprise data and trained models. AFMS can predict the user models to be applied on each test file and can facilitate a very convenient trade off between model correctness and testing speed. This enables implementing our system even for large and active enterprise shares using the computational resources of only a single household machine. 

%% The metadata and collaboration based features used in our system exhibit significant inter-dependence 
%% %For example knowing that the folder feature value corresponding to a folder in the file system is 1, one can conclude that the folder features corresponds to its ancenstor folders would also be 1. Such inter-dependence 
%% which could be leveraged to reduce the dimensionality of our data and improve the efficiency and efficacy of our system. In addition, during training, our system gives a uniform importance to training instances regardless of their recency. Since recent user activities are more reflective of current access patterns, weighing schemes for training instances may be investigated in future. 
%% Lastly, directed \textit{producer-consumer} relationships between users may be infered from past activity that can recommend content to a user (consumer) if it is created by other user (producer). 


%%% COMMENTED -- OLD CONTENT 
\comment{

\section{\uppercase{Future work}}
\label{sec:futurework}




The metadata features show a high degree of sparsity as is typical for
datasets of this size and scope. A file typically has very few
keywords in its path, and thus most of its token features would be
zero. Similarly, very few of its folder features, and at the maximum
of one extension feature of a file are non-zero.

While sparsity can be helpful for training user
models~{\cite{ngiam2011sparse}}, the large dimensionality of data may
negatively affect the performance of the models. It is possible that
the correctness and speed of the proposed system can be further
improved by capturing the interdependence between different features
through dimensionality reduction techniques such as Principal
Component
Analysis~\cite{jolliffe2005principal}\cite{van2009dimensionality}.
For example the folder features demonstrate substantial
interdependence and redundancy and techniques to transform them to a
suitable space may be explored.  A different approach would be to use
feature hashing \cite{FIXME} to reduce sparsity without detrimental
effects on performance.

Modeling the file metadata and user features in context of temporal
nature of file accesses could also be a potential direction for
further work. For example, giving more importance to recent events
while training user models may accommodate shifts in user interests,
leading to improved performance.  Identifying and modeling repetitive
activity may also be informative since users may be interested in
similar tasks after fixed time intervals, such as on the same day each
week.  As mentioned in Section~{\ref{sec:realdeployment}}, deployment
of the proposed system in an enterprise environment offers an online
framework to evaluate the trained models. Online model training or
update techniques can be developed that utilize the model evaluation
information to improve the trained models by adapting them to new
access patterns or newly observed features. For example, consider a
scenario where a trained model is seen to perform poorly because most
of the recent activity for a user is confined to a recently created
folder that was not part of the folder features in the trained
model. Such information can be derived from the online evaluation and
can be used to update features of the trained model and to adapt the
model to reflect the updated access patterns.\\
\indent In addition to training personalized user models, insight into
directed preferences of users may be useful for recommending
content. For example if it is determined that user $u_1$ often
accesses documents created by user $u_2$, then a recent modification
by $u_2$ may be useful information for $u_1$ and can be used as an
indicator to recommend relevant content.  \\
\indent Lastly, file access
prediction offers interesting possibilities for applications such as
information security, by offering new measures of access improbability.
}
%%% COMMENTED -- OLD CONTENT 
%%% COMMENTED -- OLD CONTENT 
\comment{
This paper presents a system that provides file recommendation to
assist knowledge workers process increasing volumes of
data.  The system utilizes natural language processing to derive
usable information from file metadata, and machine learning to train
personalized user models that have good predictive value, even for files that 
have not been observed in the past. Through extensive
experiments on real world data we demonstrate the feasibility of the
system to offer high quality recommendations, which is
reflected particularly in the significant recall at high precision
across eight shares. We also show that for shares exhibiting a high
degree of collaboration between its users, the predictions from
different user models can be combined to improve the performance of an
individual user's model.  It is observed that the trained models have
a high temporal longevity, and experience moderate performance
degradation for short training periods. Since the system
requires training personalized models for each user under
consideration, it should be applied only on shares
and users that display sufficient activity and are determined to be of
interest.
}
