\section{Summary}
\label{sec:conclusion}

This chapter presents a system to assist knowledge workers in
discovering useful new or modified content.  The system builds
personalized user models using features derived from file metadata and
user collaboration.  Our experiments showed that the basic modeling
approach does not scale well under heavy workloads. To address this
problem, we developed a novel optimization technique that improves
runtime performance without sacrificing prediction accuracy.  We also
showed how the technique can be adapted to improve the performance
significantly at marginal cost of the predictive correctness.

For future work, we could improve our system along different
directions.  Leveraging interactions observed between the features, we
could reduce dimensionality of the data for better efficiency and
effectiveness.  We could implement a weighing scheme that gives more
importance to the recent activity as it is more reflective of the
current preferences of users.  Finally, we could incorporate features
that capture producer-consumer relationship between users.

%% This paper presented a system that can assist knowledge workers to cope up with the growing enterprise data by recommending relevant new or modified content. The system trains file metadata models that can also leverage user collaboration to make more accurate predictions. 
%% %This paper presents a file recommender system that recommends relevant new or modified content to knowledge workers to assist them to cope up with the tremendous growth rate of enterprise data. The system uses file metadata and past user activity to train personalized user models and it is shown that collaboration demonstrated by the users can also be leveraged to make models more accurate. 
%% While experiments based on real enterprise activity demonstrated the feasibility of using the proposed system to recommend new content, it was observed that high testing computational requirement poses challenges for its deployment in enterprises, especially on large and active shares. In order to address this shortcoming, we proposed a novel model selection approach AFMS which is designed to leverage properties of the machine learning models used in our system, and of the enterprise data and trained models. AFMS can predict the user models to be applied on each test file and can facilitate a very convenient trade off between model correctness and testing speed. This enables implementing our system even for large and active enterprise shares using the computational resources of only a single household machine. 

%% The metadata and collaboration based features used in our system exhibit significant inter-dependence 
%% %For example knowing that the folder feature value corresponding to a folder in the file system is 1, one can conclude that the folder features corresponds to its ancenstor folders would also be 1. Such inter-dependence 
%% which could be leveraged to reduce the dimensionality of our data and improve the efficiency and efficacy of our system. In addition, during training, our system gives a uniform importance to training instances regardless of their recency. Since recent user activities are more reflective of current access patterns, weighing schemes for training instances may be investigated in future. 
%% Lastly, directed \textit{producer-consumer} relationships between users may be infered from past activity that can recommend content to a user (consumer) if it is created by other user (producer). 




\section{Acknowledgements}
\label{sec:ack}
We thank the anonymous ICEIS 2015 and IEEE Access reviewers for their feedback and comments on the work. We are thankful to the researchers at Symantec Research Labs for their insightful discussions and comments on the work from an early stage. The research was primarily conducted when the dissertation author was an intern at Symantec Research Labs. 

Chapter 4, in part, contains material as it appears in the Proceedings of the 17th International Conference on Enterprise Information Systems (ICEIS '15). ``Access Prediction for Knowledge Workers in Enterprise Data Repositories''. Chetan Verma, Sandeep Bhatkar, Michael Hart, Aleatha Parker-Wood, Sujit Dey. The dissertation author was the primary investigator and author of this paper. 

Chapter 4, in part, contains material as it has been accepted for publication for the IEEE Access. ``Improving Scalability of Personalized Recommendation Systems for Enterprise Knowledge Workers''. Chetan Verma, Sandeep Bhatkar, Michael Hart, Aleatha Parker-Wood, Sujit Dey. The dissertation author was the primary investigator and author of this paper. 
